% Options for packages loaded elsewhere
% Options for packages loaded elsewhere
\PassOptionsToPackage{unicode}{hyperref}
\PassOptionsToPackage{hyphens}{url}
\PassOptionsToPackage{dvipsnames,svgnames,x11names}{xcolor}
%
\documentclass[
  letterpaper,
  DIV=11,
  numbers=noendperiod]{scrreprt}
\usepackage{xcolor}
\usepackage{amsmath,amssymb}
\setcounter{secnumdepth}{5}
\usepackage{iftex}
\ifPDFTeX
  \usepackage[T1]{fontenc}
  \usepackage[utf8]{inputenc}
  \usepackage{textcomp} % provide euro and other symbols
\else % if luatex or xetex
  \usepackage{unicode-math} % this also loads fontspec
  \defaultfontfeatures{Scale=MatchLowercase}
  \defaultfontfeatures[\rmfamily]{Ligatures=TeX,Scale=1}
\fi
\usepackage{lmodern}
\ifPDFTeX\else
  % xetex/luatex font selection
\fi
% Use upquote if available, for straight quotes in verbatim environments
\IfFileExists{upquote.sty}{\usepackage{upquote}}{}
\IfFileExists{microtype.sty}{% use microtype if available
  \usepackage[]{microtype}
  \UseMicrotypeSet[protrusion]{basicmath} % disable protrusion for tt fonts
}{}
\makeatletter
\@ifundefined{KOMAClassName}{% if non-KOMA class
  \IfFileExists{parskip.sty}{%
    \usepackage{parskip}
  }{% else
    \setlength{\parindent}{0pt}
    \setlength{\parskip}{6pt plus 2pt minus 1pt}}
}{% if KOMA class
  \KOMAoptions{parskip=half}}
\makeatother
% Make \paragraph and \subparagraph free-standing
\makeatletter
\ifx\paragraph\undefined\else
  \let\oldparagraph\paragraph
  \renewcommand{\paragraph}{
    \@ifstar
      \xxxParagraphStar
      \xxxParagraphNoStar
  }
  \newcommand{\xxxParagraphStar}[1]{\oldparagraph*{#1}\mbox{}}
  \newcommand{\xxxParagraphNoStar}[1]{\oldparagraph{#1}\mbox{}}
\fi
\ifx\subparagraph\undefined\else
  \let\oldsubparagraph\subparagraph
  \renewcommand{\subparagraph}{
    \@ifstar
      \xxxSubParagraphStar
      \xxxSubParagraphNoStar
  }
  \newcommand{\xxxSubParagraphStar}[1]{\oldsubparagraph*{#1}\mbox{}}
  \newcommand{\xxxSubParagraphNoStar}[1]{\oldsubparagraph{#1}\mbox{}}
\fi
\makeatother


\usepackage{longtable,booktabs,array}
\usepackage{calc} % for calculating minipage widths
% Correct order of tables after \paragraph or \subparagraph
\usepackage{etoolbox}
\makeatletter
\patchcmd\longtable{\par}{\if@noskipsec\mbox{}\fi\par}{}{}
\makeatother
% Allow footnotes in longtable head/foot
\IfFileExists{footnotehyper.sty}{\usepackage{footnotehyper}}{\usepackage{footnote}}
\makesavenoteenv{longtable}
\usepackage{graphicx}
\makeatletter
\newsavebox\pandoc@box
\newcommand*\pandocbounded[1]{% scales image to fit in text height/width
  \sbox\pandoc@box{#1}%
  \Gscale@div\@tempa{\textheight}{\dimexpr\ht\pandoc@box+\dp\pandoc@box\relax}%
  \Gscale@div\@tempb{\linewidth}{\wd\pandoc@box}%
  \ifdim\@tempb\p@<\@tempa\p@\let\@tempa\@tempb\fi% select the smaller of both
  \ifdim\@tempa\p@<\p@\scalebox{\@tempa}{\usebox\pandoc@box}%
  \else\usebox{\pandoc@box}%
  \fi%
}
% Set default figure placement to htbp
\def\fps@figure{htbp}
\makeatother





\setlength{\emergencystretch}{3em} % prevent overfull lines

\providecommand{\tightlist}{%
  \setlength{\itemsep}{0pt}\setlength{\parskip}{0pt}}



 


\KOMAoption{captions}{tableheading}
\makeatletter
\@ifpackageloaded{bookmark}{}{\usepackage{bookmark}}
\makeatother
\makeatletter
\@ifpackageloaded{caption}{}{\usepackage{caption}}
\AtBeginDocument{%
\ifdefined\contentsname
  \renewcommand*\contentsname{Table of contents}
\else
  \newcommand\contentsname{Table of contents}
\fi
\ifdefined\listfigurename
  \renewcommand*\listfigurename{List of Figures}
\else
  \newcommand\listfigurename{List of Figures}
\fi
\ifdefined\listtablename
  \renewcommand*\listtablename{List of Tables}
\else
  \newcommand\listtablename{List of Tables}
\fi
\ifdefined\figurename
  \renewcommand*\figurename{Figure}
\else
  \newcommand\figurename{Figure}
\fi
\ifdefined\tablename
  \renewcommand*\tablename{Table}
\else
  \newcommand\tablename{Table}
\fi
}
\@ifpackageloaded{float}{}{\usepackage{float}}
\floatstyle{ruled}
\@ifundefined{c@chapter}{\newfloat{codelisting}{h}{lop}}{\newfloat{codelisting}{h}{lop}[chapter]}
\floatname{codelisting}{Listing}
\newcommand*\listoflistings{\listof{codelisting}{List of Listings}}
\makeatother
\makeatletter
\makeatother
\makeatletter
\@ifpackageloaded{caption}{}{\usepackage{caption}}
\@ifpackageloaded{subcaption}{}{\usepackage{subcaption}}
\makeatother
\usepackage{bookmark}
\IfFileExists{xurl.sty}{\usepackage{xurl}}{} % add URL line breaks if available
\urlstyle{same}
\hypersetup{
  pdftitle={Muhammad Kinan Arkansyaddad},
  pdfauthor={13523152 Muhammad Kinan Arkansyaddad},
  colorlinks=true,
  linkcolor={blue},
  filecolor={Maroon},
  citecolor={Blue},
  urlcolor={Blue},
  pdfcreator={LaTeX via pandoc}}


\title{Muhammad Kinan Arkansyaddad}
\usepackage{etoolbox}
\makeatletter
\providecommand{\subtitle}[1]{% add subtitle to \maketitle
  \apptocmd{\@title}{\par {\large #1 \par}}{}{}
}
\makeatother
\subtitle{Portfolio Asesmen II-2100 KIPP}
\author{13523152 Muhammad Kinan Arkansyaddad}
\date{2025-10-31}
\begin{document}
\maketitle

\renewcommand*\contentsname{Table of contents}
{
\hypersetup{linkcolor=}
\setcounter{tocdepth}{2}
\tableofcontents
}

\bookmarksetup{startatroot}

\chapter*{\texorpdfstring{Kinan Codex: \nChapter II2100 Interpersonal
and Public
Communication}{Kinan Codex: II2100 Interpersonal and Public Communication}}\label{kinan-codex-ii2100-interpersonal-and-public-communication}
\addcontentsline{toc}{chapter}{Kinan Codex: \nChapter II2100
Interpersonal and Public Communication}

\markboth{Kinan Codex: \nChapter II2100 Interpersonal and Public
Communication}{Kinan Codex: \nChapter II2100 Interpersonal and Public
Communication}

\begin{figure}[H]

{\centering \includegraphics[width=9.5\linewidth,height=\textheight,keepaspectratio]{images/foto_1.jpg}

}

\caption{It's me!!}

\end{figure}%

\textbf{PRESS {[}START{]} TO PLAY THE GAME!}

\bookmarksetup{startatroot}

\chapter{UTS-1 All About Me}\label{uts-1-all-about-me}

The Journey So Far\ldots{}

\hfill\break

\section{The Player Has Entered The
Game}\label{the-player-has-entered-the-game}

Jika hidup adalah sebuah game\ldots{} perkenalkan, ini `Player
Character' saya. Saat ini, saya sedang berada di \textbf{`Level 3'}
perkuliahan.

\begin{itemize}
\tightlist
\item
  `Tutorial Stage' (Tingkat 1 \& 2) sudah lewat.
\item
  \emph{Difficulty}-nya? Sudah bukan \textbf{`Hard'} lagi, tetapi sudah
  bagai difficulty \textbf{`Dante Must Die'} di game \emph{Devil May Cry
  Series}.
\item
  \emph{NPC} (Dosen) punya \emph{quest-line} yang makin rumit, dan
  \emph{`EXP'} (ilmu) yang dibutuhkan untuk naik level terasa makin
  banyak.
\end{itemize}

\begin{center}\rule{0.5\linewidth}{0.5pt}\end{center}

\section{Character Sheet: Kinan The Tired
Student}\label{character-sheet-kinan-the-tired-student}

\subsection{\texorpdfstring{\textbf{Informasi
Dasar}}{Informasi Dasar}}\label{informasi-dasar}

\begin{itemize}
\tightlist
\item
  \textbf{Title:} The Tired Student
\item
  \textbf{Main Class:} Game Summoner (Game Dev)
\item
  \textbf{Sub Class:} Red Devil Loyalist (GGMU!)
\item
  \textbf{Guild:} HMIF ITB \& GIM ITB
\item
  \textbf{Origin (World):} The Heart of the Mountains, Bandung
\item
  \textbf{Level:} 3 (Tingkat 3)
\end{itemize}

\subsection{\texorpdfstring{\textbf{Base
Stats}}{Base Stats}}\label{base-stats}

\begin{itemize}
\tightlist
\item
  \textbf{INT (Intellect):} 15 (Cukup untuk \emph{coding}, tapi butuh
  \emph{potion} `Kopi')
\item
  \textbf{DEX (Dexterity):} 10 (Kecepatan ngetik di keyboard)
\item
  \textbf{STA (Stamina):} 7 (\emph{Debuff} `Begadang' kronis)
\item
  \textbf{SOC (Social):} 10 (Dapat \emph{buff} setelah join `Guild')
\item
  \textbf{LUK (Luck):} 8 (Apa itu Bug? Maksudmu'Fitur' yang tak
  disengaja?)
\end{itemize}

\begin{center}\rule{0.5\linewidth}{0.5pt}\end{center}

\section{\texorpdfstring{Job Skills:
\texttt{Game\ Summoner}}{Job Skills: Game Summoner}}\label{job-skills-game-summoner}

Ini adalah \emph{Skill Tree} utama saya, tempat saya menghabiskan
sebagian besar \emph{EXP} (Waktu \& Usaha).

\subsection{\texorpdfstring{\textbf{Active Skills
(Teknikal)}}{Active Skills (Teknikal)}}\label{active-skills-teknikal}

\begin{itemize}
\tightlist
\item
  \texttt{Code\ Weaving\ (Lv.\ 2)}: Merapal \emph{script} GDScript untuk
  menghidupkan \emph{object}.
\item
  \texttt{World\ Crafting\ (Lv.\ 1)}: Mendesain \emph{environment} \&
  \emph{level} dasar di Godot Engine.
\item
  \texttt{Bug\ Squashing\ (Lv.\ 2)}: \emph{Skill} \emph{debuff} untuk
  melacak dan membasmi \emph{bug} (kadang butuh 3-4 kali \emph{cast}).
\item
  \texttt{Pixel\ Art\ (Lv.\ 1)}: Kemampuan \emph{crafting} aset visual
  3D.
\end{itemize}

\subsection{\texorpdfstring{\textbf{Passive Skills
(Konseptual)}}{Passive Skills (Konseptual)}}\label{passive-skills-konseptual}

\begin{itemize}
\tightlist
\item
  \texttt{Gamer\textquotesingle{}s\ Intuition}: Memahami \emph{game
  loop} \& \emph{player experience} secara alami.
\item
  \texttt{Storyteller\textquotesingle{}s\ Mind}: Selalu mencari narasi
  di balik sebuah mekanik.
\end{itemize}

\begin{center}\rule{0.5\linewidth}{0.5pt}\end{center}

\section{\texorpdfstring{Sub-Job Skills:
\texttt{Red\ Devil\ Loyalist}}{Sub-Job Skills: Red Devil Loyalist}}\label{sub-job-skills-red-devil-loyalist}

Meskipun fokus utama saya \emph{crafting}, `jiwa' saya juga ada di
lapangan hijau. Ini adalah \emph{skill tree} sekunder saya.

\begin{itemize}
\tightlist
\item
  \texttt{Tactical\ Analysis\ (Passive)}: Kemampuan menganalisis
  formasi, strategi, dan \emph{meta} (di Football Manager dan FIFA(FC)).
\item
  \texttt{Chant\ of\ Allegiance\ (Active)}: \emph{Skill} \emph{buff}
  moral. Aktif saat nonton bareng. Memberi \emph{support} pada tim meski
  sedang kalah 0-2 di menit 80.
\item
  \texttt{Historical\ Knowledge\ (Passive)}: Mengingat \emph{lore}
  (sejarah) Manchester United lebih baik daripada \emph{lore} mata
  kuliah.
\item
  \texttt{Resilience\ (Trait)}: \emph{Trait} unik yang didapat dari
  \emph{Sub-Class} ini. Sudah terbiasa dengan `Harapan Palsu' dan `Epic
  Comeback'. Sangat berguna saat \emph{project game} berkali-kali
  \emph{crash}.
\end{itemize}

\begin{center}\rule{0.5\linewidth}{0.5pt}\end{center}

\section{Equipment \& Traits}\label{equipment-traits}

Sebuah \emph{character} tidak lengkap tanpa \emph{gear} dan
\emph{traits} unik.

\subsection{\texorpdfstring{\textbf{Current
Equipment}}{Current Equipment}}\label{current-equipment}

\begin{itemize}
\tightlist
\item
  \textbf{Main Hand:} \texttt{The\ Grimoire\ of\ Code} (Lenovo Ideapad
  Gaming 3i)
\item
  \textbf{Off-Hand:} \texttt{Rhythm\ Controllers} (Noir N1 Pro)
\item
  \textbf{Helm:} \texttt{Focus\ Charm} (Kinera Wyvern (IEM))
\item
  \textbf{Armor:}
  \texttt{Jersey\ \textquotesingle{}Setan\ Merah\textquotesingle{}}
  (Memberi +5 WLP saat dipakai \emph{coding})
\end{itemize}

\subsection{\texorpdfstring{\textbf{Unique Traits
(Sifat)}}{Unique Traits (Sifat)}}\label{unique-traits-sifat}

\begin{itemize}
\tightlist
\item
  \textbf{`Dual Focus'}: Mampu beralih \emph{context} antara
  menganalisis \emph{bug} di kode dan menganalisis \emph{blunder} di
  lini pertahanan.
\item
  \textbf{`Process Enjoyer'}: Memahami bahwa \emph{`The Rebuild'} (baik
  di tim atau di \emph{project}) adalah bagian dari \emph{journey} yang
  panjang.
\item
  \textbf{`Creative Spark'}: Ketertarikan pada bidang lain (sepak bola,
  Pop Culture) sering memberi ide tak terduga untuk mekanik game.
\end{itemize}

\begin{center}\rule{0.5\linewidth}{0.5pt}\end{center}

\section{Backstory: The Tutorial
Years}\label{backstory-the-tutorial-years}

Setiap `player' punya `origin story'. Cerita saya dimulai bukan dengan
takdir hebat, tapi di sebuah kota kecil berkabut bernama Inaba.

Itu terjadi di puncak era sekolah online. Dunia nyata terasa
\emph{offline}, terisolasi, dan abu-abu. Lalu saya `jatuh' ke dunia TV
dalam \textbf{Persona 4 Golden}.

Itu bukan sekadar `main'. Itu adalah `hidup'. Saya menghabiskan 70+ jam
dalam 5 hari. Sampai saat ini, itu adalah game terbaik yang pernah saya
mainkan.

\begin{center}\rule{0.5\linewidth}{0.5pt}\end{center}

\section{Backstory: Reaching Out To The
Truth}\label{backstory-reaching-out-to-the-truth}

Di saat dunia nyata `terjeda', \emph{Persona 4 Golden} memberi saya
`rutinitas'. Saya bersekolah, membangun pertemanan (\emph{Social
Links}), dan melawan `monster'.

\emph{Quest} saya pun berubah. Bukan lagi ``Bagaimana cara benda ini
bekerja?'' tapi\ldots{}

\textbf{``Siapa diri saya yang sebenarnya, dan siapa orang-orang yang
berarti bagi saya?''}

\textbf{Wawasan (Insight):} P4G punya satu pesan inti: \textbf{Kekuatan
sejati datang dari menerima `Shadow Self' kita}---bagian dari diri kita
yang kita benci, sembunyikan, atau takuti---dan \textbf{membangun ikatan
tulus (\emph{Bonds})} dengan orang lain.

--

\section{Backstory: The Path to `The
Creator'}\label{backstory-the-path-to-the-creator}

Saya bermain game tentang \emph{koneksi manusia} yang mendalam, justru
di saat saya paling \emph{terisolasi} dari dunia nyata.

\emph{Persona 4 Golden} adalah `Tutorial' sempurna untuk perjalanan
menuju kuliah: 1. Ia mengajarkan saya bahwa masa transisi (pindah
sekolah/kuliah) adalah tentang \textbf{keberanian menghadapi `Shadow'
kita} (rasa takut, \emph{imposter syndrome}). 2. Ia membuktikan bahwa
`Stats' (nilai) tidak ada artinya tanpa \textbf{`Social Links'}
(pertemanan dan jaringan).

Saya tidak hanya menyelesaikan game itu. Saya `hidup' di dalamnya.

Dan saya sadar, saya bukan cuma mau jadi `Player'. Saya mau jadi
`Creator'. Saya ingin membuat `dunia' yang bisa memberi orang lain
`ikatan' dan `makna' sekuat yang saya rasakan di Inaba.

Itulah `Class' Teknik Informatika (Game Development) yang saya pilih.

\begin{center}\rule{0.5\linewidth}{0.5pt}\end{center}

\section{Main Quest: The Fog of War}\label{main-quest-the-fog-of-war}

Di Level 1 dan 2, \emph{Main Quest Log} saya jelas: ``Selesaikan Kelas
A,'' ``Lulus Ujian B.'' Jalurnya lurus, seperti \emph{tutorial level}
yang memegang tangan kita.

Sekarang di Level 3, semuanya berubah.

\emph{Map} dunia tiba-tiba terbuka lebar. \emph{Fog of War} (kabut)
menutupi sebagian besar \emph{quest area} bernama `Masa Depan'.
\emph{Quest marker} tidak lagi menunjuk ke satu titik, tapi berkedip di
mana-mana.

\emph{Main Quest} saya pun \emph{di-update}. Judulnya bukan lagi
`Menaklukkan Musuh', tapi: \textbf{{[}SURVIVE{]}} dan \textbf{{[}FIND
THE PATH{]}}.

\begin{center}\rule{0.5\linewidth}{0.5pt}\end{center}

\section{Main Quest: The `Adulthood'
Debuff}\label{main-quest-the-adulthood-debuff}

Ini adalah \emph{quest} yang paling sulit, karena \emph{boss}-nya tidak
terlihat. \emph{Boss}-nya adalah \emph{debuff} (status negatif) permanen
bernama `Overthinking'.

Setiap malam, ada \emph{cutscene} internal yang berputar: * ``Ambil
\emph{Job Class} (spesialisasi) apa?'' * ``Masuk \emph{Guild} (tempat
magang) yang mana?'' * ``Bagaimana jika saya salah menaikkan \emph{Skill
Tree}?''

Ini adalah \emph{quest} bernama `Menjadi Dewasa'. Pilihan saya tiba-tiba
punya `Weight' (bobot). \emph{Save file} tidak bisa di-\emph{load} ulang
dengan mudah.

\textbf{Wawasan (Insight):} \emph{Quest} ini mengajarkan saya bahwa
`Survive' bukan berarti `diam di tempat'. `Survive' berarti terus
berjalan maju, meski ke dalam kabut.

\emph{Objective} saya sekarang adalah: Berhenti mencari `Walkthrough'
yang sempurna, dan mulai percaya pada \emph{character build} yang sudah
saya latih 2 tahun ini untuk mengambil langkah pertama.

\begin{center}\rule{0.5\linewidth}{0.5pt}\end{center}

\section{Side Quests \& `Mana Regen'}\label{side-quests-mana-regen}

Sebuah `Player' tidak bisa \emph{grinding} 24/7.

`Main Quest' Level 3 itu sangat menguras `Stamina' dan `MP' (Mental
Points). Jika saya terus-menerus \emph{grinding} tanpa istirahat, saya
akan terkena \emph{status effect} `Burnout'.

Jadi, ini adalah \emph{Side Quests} yang saya lakukan untuk
\emph{recovery}---untuk mengisi ulang `Mana' saya.

\begin{center}\rule{0.5\linewidth}{0.5pt}\end{center}

\section{Side Quest 1: `The Theatre of
Dreams'}\label{side-quest-1-the-theatre-of-dreams}

Ini adalah \emph{quest} mingguan yang paling penting: menjalankan
\emph{Sub-Job} saya sebagai \texttt{Red\ Devil\ Loyalist}.

\begin{itemize}
\tightlist
\item
  \textbf{Objective:} Menonton Manchester United.
\item
  \textbf{Wawasan (Insight):} Ini adalah ritual. Kadang, \emph{quest}
  ini malah menambah stres (tergantung hasil), tapi ini adalah
  `pelarian' yang saya pilih.
\item
  \textbf{Reward:} Ini melatih \emph{stat} pasif saya yang paling
  penting: \textbf{\texttt{Resilience} (Ketahanan)}.
\end{itemize}

Bertahun-tahun mendukung tim ini telah menempa \texttt{Willpower} saya.
Saya jadi terbiasa dengan `The Rebuild', `Epic Comeback', dan `Harapan
Palsu'.

\emph{Trait} `Resilience' ini sangat berguna saat \emph{project game}
saya \emph{crash} H-1 \emph{deadline}. Saya sudah terbiasa.

\begin{center}\rule{0.5\linewidth}{0.5pt}\end{center}

\section{Side Quest 2: `Riset' (a.k.a Main Game
Lain)}\label{side-quest-2-riset-a.k.a-main-game-lain}

\begin{itemize}
\tightlist
\item
  \textbf{Objective:} Menjelajahi \emph{world} lain yang sudah jadi.
\item
  \textbf{Wawasan (Insight):} Ini bukan lari dari tanggung jawab. Saya
  menyebutnya `Riset Pasar'.
\item
  \textbf{Reward:} `Inspiration'.
\end{itemize}

Melihat \emph{game design} orang lain, mempelajari \emph{storytelling}
mereka, atau sekadar menikmati \emph{gameplay}---semua itu memberi saya
ide-ide baru.

Ini mengingatkan saya \emph{mengapa} `Main Quest' (Game Dev) saya sangat
sepadan: karena saya ingin suatu hari nanti menciptakan `dunia' yang
sama berkesannya bagi orang lain.

\begin{center}\rule{0.5\linewidth}{0.5pt}\end{center}

\section{Side Quest 3: `Touch Grass'}\label{side-quest-3-touch-grass}

\begin{itemize}
\tightlist
\item
  \textbf{Objective:} Keluar dari \emph{dungeon} (kamar) dan bertemu
  `NPC' di dunia nyata.
\item
  \textbf{Wawasan (Insight):} Berdiskusi dengan teman (\emph{party
  members}) di `Guild Hall' {[}Organisasi/Himpunan{]} atau sekadar
  nongkrong.
\item
  \textbf{Reward:} `Social Buff'.
\end{itemize}

Ini adalah pengingat terpenting: \emph{Quest} `Adulthood' ini, meskipun
terasa personal dan sepi, sebenarnya adalah \emph{Massive Multiplayer
Online (MMO) quest}.

Saya tidak sendirian dalam \emph{grinding} ini.

\begin{center}\rule{0.5\linewidth}{0.5pt}\end{center}

\section{The Character Build (So Far)}\label{the-character-build-so-far}

Jadi, inilah `All About Me' di Level 3.

Saya adalah seorang \texttt{Game\ Summoner\ (Trainee)}\ldots{}
\ldots yang `Origin Story'-nya dibentuk oleh makna `Social Links' dalam
\emph{Persona 4 Golden} di tengah isolasi. \ldots yang `Main Quest'-nya
saat ini adalah \textbf{{[}SURVIVE{]}} dan menavigasi `Kabut'
\emph{overthinking} tentang masa depan. \ldots dan yang `Mana'-nya
di-regen oleh \emph{sub-job} \texttt{Red\ Devil\ Loyalist} yang
mengajarkan \texttt{Resilience} abadi.

\emph{Character build} saya masih jauh dari `meta'. Masih banyak
\emph{bug} yang harus di-\emph{fix} dan \emph{skill} yang harus
di-\emph{grind}.

\bookmarksetup{startatroot}

\chapter{UTS-2 My Songs for You}\label{uts-2-my-songs-for-you}

\url{https://youtu.be/6Oib9lkWJeU?si=TwjskBdK7A44nGFo}

\textbf{Pursuing My True Self from Persona 4 Golden}

We are living our lives Abound with so much information

Come on, let go of the remote Don't you know you're letting all the junk
flood in I try to stop the flow, double-clicking on the go But it's no
use, hey I'm being consumed

Loading\ldots{} Loading\ldots{} Loading\ldots{} Quickly reaching maximum
capacity Warning\ldots{} Warning\ldots{} Warning\ldots{} Gonna
short-circuit my identity (ahh)

Get up on your feet, tear down the walls Catch a glimpse of the hollow
world Snooping 'round town will get you nowhere You're locked up in your
mind

We're all trapped in a maze of relationships Life goes on with or
without you I swim in the sea of the unconscious I search for your
heart, pursuing my true self

\bookmarksetup{startatroot}

\chapter{UTS-3: My Stories for You}\label{uts-3-my-stories-for-you}

Today Is Always Your Best Day 🍀

\hfill\break

\section{Membuka `Quest Log'
Inspirasi}\label{membuka-quest-log-inspirasi}

Sebagai seorang \emph{player} di `Level 3' (Tingkat 3), saya sadar hidup
ini seperti memainkan two \emph{game modes} sekaligus.

Dan saya ingin membagikan dua ``Game Guide'' atau ``Walkthrough'' yang
menurut saya paling inspiratif untuk kita mainkan.

\textbf{Mode \#1:} JRPG Kompleks tentang \emph{Masa Depan}. \textbf{Mode
\#2:} \emph{Slice-of-Life Minigame} tentang \emph{Hari Ini}.

\begin{center}\rule{0.5\linewidth}{0.5pt}\end{center}

\section{Game Guide \#1: `Persona 4' (The Main
Quest)}\label{game-guide-1-persona-4-the-main-quest}

Seperti yang pernah saya ceritakan, \emph{Persona 4} adalah `Game of My
Life'.

Ini adalah \emph{game} yang sempurna untuk `Level 3' kita. Ini adalah
JRPG yang \emph{gameplay}-nya adalah tentang: * Berjalan menembus `Fog
of War' (Kabut ketidakpastian masa depan). * Melawan `Shadow Self' (Si
`Overthinking', si `Imposter Syndrome', si `Tired Student'). * Dan
satu-satunya cara untuk menang adalah dengan menaikkan \emph{stats}
`Social Links' (Ikatan dengan orang lain).

\begin{center}\rule{0.5\linewidth}{0.5pt}\end{center}

\section{`Persona 4': Lore \& Wawasan}\label{persona-4-lore-wawasan}

\textbf{Kisah Inspiratif \#1:} ``Kita tidak bisa lari dari `Shadow'
kita.''

\emph{Game} ini mengajarkan saya bahwa semua `Overthinking' dan
`Kecemasan' (Shadow Self) itu tidak bisa dikalahkan sendirian.

Satu-satunya cara untuk `menang' dalam \emph{quest} `Menjadi Dewasa'
adalah dengan \textbf{``Reaching Out to the Truth''}---mempercayai
`Party Members' kita (teman-teman, relasi intim) untuk menghadapi kabut
bersama.

Ini adalah \emph{game guide} tentang \textbf{MENGAPA} kita berjuang:
Demi ikatan dan masa depan.

\begin{center}\rule{0.5\linewidth}{0.5pt}\end{center}

\section{Game Guide \#2: `Yotsuba\&!' (The Daily
Quest)}\label{game-guide-2-yotsuba-the-daily-quest}

Tapi, ada `game guide' kedua yang sama pentingnya. Ini bukan JRPG yang
berat. Ini adalah \emph{manga}.

\emph{Manga} ini tidak punya `Main Quest' yang rumit. Tidak ada `Shadow
Self'. Tidak ada `Fog of War'.

\emph{Boss}-nya adalah ``Hari Hujan''. \emph{Quest}-nya adalah ``Pergi
Membeli Susu''. \emph{Stats}-nya hanya satu: \textbf{`ENJOY EVERYTHING'
(Menikmati Segalanya)}, dan Yotsuba (tokoh utamanya) punya \emph{stat}
ini di level MAX.

\begin{figure}[H]

{\centering \includegraphics[width=9.5\linewidth,height=\textheight,keepaspectratio]{My_Stories_for_You/../images/yotsuba.jpg}

}

\caption{Yotsuba \& Her Father}

\end{figure}%

\begin{center}\rule{0.5\linewidth}{0.5pt}\end{center}

\section{`Yotsuba\&!': Lore \& Wawasan}\label{yotsuba-lore-wawasan}

\textbf{Kisah Inspiratif \#2:} ``Semua hal adalah `Play'.''

\emph{Game} ini mengajarkan saya sesuatu yang dilupakan oleh
\emph{Persona 4}: Yotsuba mengubah \emph{game loop} yang paling
membosankan (`the grind') menjadi \emph{gameplay} yang paling seru.

Dia menemukan `fun' dalam \emph{daily quest} yang kita anggap remeh:
mencoba AC baru, menangkap serangga, atau sekadar makan es krim.

Ini adalah \emph{game guide} tentang \textbf{BAGAIMANA} kita harus
menjalani perjuangan itu: Dengan menemukan kebahagiaan di level paling
dasar.

\begin{figure}[H]

{\centering \pandocbounded{\includegraphics[keepaspectratio]{My_Stories_for_You/../images/image.png}}

}

\caption{Yotsuba}

\end{figure}%

\begin{center}\rule{0.5\linewidth}{0.5pt}\end{center}

\section{More Resources You Can
Watch}\label{more-resources-you-can-watch}

\url{https://youtu.be/CujlchNaHNc?si=M87U_XedHrkg10M0}

\url{https://youtu.be/YOjw9KNKgLU?si=Ez-VNINOj9fQ3gmj}

\url{https://youtu.be/qiNOdt8JcNs?si=3RQo3ZSeF_lkYZhA}

\url{https://youtu.be/WjmzycUOeO8?si=RaQ8LZveMr3ITqYk}

\begin{center}\rule{0.5\linewidth}{0.5pt}\end{center}

\section{Pesan Terakhir: `The Co-op
Strategy'}\label{pesan-terakhir-the-co-op-strategy}

Ini adalah kisah yang ingin saya bagikan kepada Anda.

Kita sering kali terlalu fokus memainkan `Game \#1' (\emph{Persona 4}).
Kita terjebak dalam \emph{quest} `Menjadi Dewasa', menaikkan
\emph{stats} `Karir', dan \emph{overthinking} soal `Kabut Masa Depan'.

Kita lupa bahwa kita juga harus memainkan `Game \#2'
(\emph{Yotsuba\&!}).

\textbf{Kisah inspiratifnya adalah:} Kita butuh `Main Quest'
\emph{Persona 4} untuk memberi kita \textbf{tujuan} dan \textbf{ikatan}.
Tapi kita butuh `Daily Quest' \emph{Yotsuba} untuk memberi kita
\textbf{kebahagiaan} hari ini.

Jangan lupa menikmati \emph{gameplay} sederhana di tengah
\emph{grinding} `Level 3' yang berat ini.

\bookmarksetup{startatroot}

\chapter{UTS-4: My Shape}\label{uts-4-my-shape}

Analisis Teknis `Character Build' Saya

\hfill\break

\section{The Player Codex: Sebuah
Laporan}\label{the-player-codex-sebuah-laporan}

Jika UTS-1, 2, \& 3 adalah \emph{lore} dan \emph{cutscene}, maka UTS-4
ini adalah \textbf{`Player Stats Screen'}.

Ini adalah laporan teknis tentang `siapa saya' berdasarkan empat
\emph{game mechanics} utama: 1. \textbf{Identitas Naratif} (Character
Lore / Backstory) 2. \textbf{Piagam Diri} (The Main Quest Log) 3.
\textbf{Asesmen VIA} (Core Stats \& Passive Skills) 4. \textbf{My SHAPE}
(The Full Character Build Analysis)

\begin{center}\rule{0.5\linewidth}{0.5pt}\end{center}

\section{Identitas Naratif (Character
Lore)}\label{identitas-naratif-character-lore}

Setiap \emph{character} punya cerita. Identitas Naratif saya adalah
\emph{story arc} yang telah saya bagikan.

\begin{itemize}
\tightlist
\item
  \textbf{The Origin Story:} \emph{Persona 4 Golden} yang mengajarkan
  pentingnya `Social Links' di tengah isolasi.
\item
  \textbf{The Current Arc:} `Level 3' (Tingkat 3) yang penuh `Fog of
  War' (overthinking) dan \emph{quest} {[}SURVIVE{]} \& {[}FIND THE
  PATH{]}.
\item
  \textbf{The Sub-Plot:} Loyalitas sebagai \texttt{Red\ Devil\ Loyalist}
  yang menanamkan \emph{stat} \texttt{Resilience}.
\item
  \textbf{The Core Philosophy:} Keseimbangan antara `Main Quest'
  (\emph{Persona 4}) dan `Daily Quest' (\emph{Yotsuba\&!}).
\end{itemize}

\begin{center}\rule{0.5\linewidth}{0.5pt}\end{center}

\section{Piagam Diri (My Main Quest
Log)}\label{piagam-diri-my-main-quest-log}

`Piagam Diri' adalah `Player's Oath' atau `Guild Charter' saya. Ini
adalah \emph{objective} utama yang saya tetapkan untuk \emph{game} ini,
berdasarkan \emph{lore} yang telah saya kumpulkan.

\begin{itemize}
\item
  \textbf{Misi Utama (Mission):} \textgreater{} ``Menjadi `Game Creator'
  (Game Summoner) yang karyanya mampu menciptakan `dunia' dan `Social
  Links' (ikatan) yang bermakna bagi para pemainnya, seperti yang saya
  rasakan di Inaba.''
\item
  \textbf{Nilai Inti (Core Values):} \textgreater{} \texttt{Resilience}
  (Mentalitas `The Rebuild' \& `Fergie Time'). \textgreater{}
  \texttt{Koneksi} (Kekuatan `Social Links' \& `Party Members').
  \textgreater{} \texttt{Kreativitas} (Kemampuan `World Crafting' \&
  `Creative Spark'). \textgreater{} \texttt{Otentisitas} (Berani `Pursue
  My True Self' \& menghadapi `Shadow'). \textgreater{}
  \texttt{Keseimbangan} (Menikmati `Daily Quest' ala Yotsuba).
\item
  \textbf{Prinsip (Guiding Principles):} \textgreater{} ``Berhenti
  mencari `Walkthrough' yang sempurna; percaya pada \emph{build} sendiri
  untuk melangkah ke `Fog of War'.'' \textgreater{} ``\emph{Quest} ini
  adalah \emph{MMO}, bukan \emph{Single-Player}. Selalu `Reach Out' ke
  `Party Members'.'' \textgreater{} ``Seimbangkan \emph{grinding} `Main
  Quest' (Persona 4) dengan menikmati \emph{gameplay} sederhana `Daily
  Quest' (Yotsuba\&!).''
\end{itemize}

\begin{center}\rule{0.5\linewidth}{0.5pt}\end{center}

\section{Asesmen VIA (Core Stats \& Passive
Skills)}\label{asesmen-via-core-stats-passive-skills}

Asesmen VIA menunjukkan \emph{Base Stats} yang paling
tinggi---\emph{skill} pasif yang aktif secara alami.

Setelah dianalisis, 2 \emph{Passive Skills} teratas saya yang paling
menonjol adalah:

\begin{enumerate}
\def\labelenumi{\arabic{enumi}.}
\tightlist
\item
  \textbf{\(\text{Creativity (Kreativitas)}\)}

  \begin{itemize}
  \tightlist
  \item
    ``Kemampuan `World Crafting'. Selalu memikirkan cara-cara baru dan
    orisinal untuk melakukan sesuatu, baik itu mendesain \emph{game}
    atau memecahkan \emph{bug}.''
  \end{itemize}
\item
  \textbf{\(\text{Humor (Humor)}\)}

  \begin{itemize}
  \tightlist
  \item
    ``Kemampuan `Morale Boost'. Suka tertawa dan membawa senyum ke
    `Party Members', menemukan humor bahkan di tengah \emph{dungeon}
    yang sulit.''
  \end{itemize}
\end{enumerate}

\begin{center}\rule{0.5\linewidth}{0.5pt}\end{center}

\section{`My SHAPE' (The Full Build
Analysis)}\label{my-shape-the-full-build-analysis}

`SHAPE' adalah \emph{framework} lengkap untuk membedah \emph{build
character} saya.

Ini adalah analisisnya:

\begin{itemize}
\tightlist
\item
  \textbf{S (Strengths / Spiritual Gifts):} \emph{Skill} Inti / `Magic'
\item
  \textbf{H (Heart / Passion):} `Motivation Core' / `Drive'
\item
  \textbf{A (Abilities):} `Active Skills' / Profisiensi
\item
  \textbf{P (Personality):} `Character Archetype'
\item
  \textbf{E (Experiences):} `EXP Log' / `Completed Quests'
\end{itemize}

\begin{center}\rule{0.5\linewidth}{0.5pt}\end{center}

\section{SHAPE: S (Strengths)}\label{shape-s-strengths}

\textbf{S - Strengths (Skill Inti)} Ini adalah \emph{skill} bawaan saya,
mirip dengan \emph{stats} VIA. * \texttt{Humor}: Kemampuan `Morale
Boost' untuk diri sendiri dan `Party Members'. *
\texttt{Creativity\ (Kreativitas)}: Kemampuan `World Crafting' \&
`Storyteller's Mind'. * \texttt{Resilience\ (Ketahanan)}: \emph{Trait}
pasif yang di-\emph{grinding} dari \emph{Sub-Class} `Red Devil
Loyalist'.

\begin{center}\rule{0.5\linewidth}{0.5pt}\end{center}

\section{SHAPE: H (Heart)}\label{shape-h-heart}

\textbf{H - Heart (Motivation Core)} Ini adalah \emph{`lore'} yang
menggerakkan saya. * \textbf{Passion:} \texttt{Game\ Development}
(Menciptakan `dunia' dan `ikatan' baru). * \textbf{Interest:}
\texttt{Sepak\ Bola} (Loyalitas, \texttt{Resilience}, dan strategi). *
\textbf{Values:} \texttt{Koneksi\ Manusia} (Pelajaran dari \emph{Persona
4}).

\begin{center}\rule{0.5\linewidth}{0.5pt}\end{center}

\section{SHAPE: A (Abilities)}\label{shape-a-abilities}

\textbf{A - Abilities (Active Skills)} Ini adalah \emph{skill} yang
sudah saya \emph{grinding} (latih). * \textbf{Teknikal:}
\texttt{Code\ Weaving\ (GDScript,\ C++,\ etc)}, \texttt{Bug\ Squashing},
\texttt{World\ Crafting\ (Godot)}. * \textbf{Non-Teknikal:}
\texttt{Critikal\ Thinker},
\texttt{Storyteller\textquotesingle{}s\ Mind}.

\begin{center}\rule{0.5\linewidth}{0.5pt}\end{center}

\section{SHAPE: P (Personality)}\label{shape-p-personality}

\textbf{P - Personality (Archetype)} \emph{Archetype} dasar dari
\emph{character} ini. * MBTI: INTP

\begin{center}\rule{0.5\linewidth}{0.5pt}\end{center}

\section{SHAPE: E (Experiences)}\label{shape-e-experiences}

Ini adalah `EXP Log'---\emph{quest-quest} penting yang telah membentuk
\emph{build} saya sejauh ini. EXP Gained dan Lesson hanyalah satu dari
banyak pengalaman yang didapat.

\begin{itemize}
\tightlist
\item
  \textbf{`Tutorial Level' (Sekolah):}

  \begin{itemize}
  \tightlist
  \item
    \emph{EXP Gained:} Menemukan \emph{Persona 4 Golden}.
  \item
    \emph{Lesson:} Belajar pentingnya `Social Links' dan `Pursuing My
    True Self'.
  \end{itemize}
\item
  \textbf{`The Guild Hall' (HMIF \& GIM ITB):}

  \begin{itemize}
  \tightlist
  \item
    \emph{EXP Gained:} Bekerja dalam `Party' (tim).
  \item
    \emph{Lesson:} \emph{Quest} ini adalah \emph{MMO}, bukan
    \emph{Single-Player}.
  \end{itemize}
\item
  \textbf{`The Forge' (Proyek Game Dev):}

  \begin{itemize}
  \tightlist
  \item
    \emph{EXP Gained:} Mengubah ide menjadi \emph{prototype}.
  \item
    \emph{Lesson:} Mengaplikasikan \texttt{Resilience} saat menghadapi
    \emph{bug}.
  \end{itemize}
\item
  \textbf{`The Overthinking Fog' (Tingkat 3):}

  \begin{itemize}
  \tightlist
  \item
    \emph{EXP Gained:} Menghadapi `Adulthood Debuff'.
  \item
    \emph{Lesson:} Belajar menyeimbangkan `Main Quest' (P4G) dengan
    `Daily Quest' (Yotsuba).
  \end{itemize}
\end{itemize}

\begin{center}\rule{0.5\linewidth}{0.5pt}\end{center}

\section{Laporan Selesai: `My Shape'}\label{laporan-selesai-my-shape}

Ini adalah \emph{build character} saya di `Level 3'.

Berdasarkan \emph{lore} (Identitas Naratif), \emph{quest log} (Piagam
Diri), \emph{stats} (VIA), dan \emph{full analysis} (SHAPE),
\emph{build} ini mungkin belum `meta' (sempurna).

Tapi \emph{build} ini \textbf{Otentik}. Dan \emph{build} ini siap untuk
\emph{level} selanjutnya.

\bookmarksetup{startatroot}

\chapter{UTS-5 My Personal Reviews}\label{uts-5-my-personal-reviews}

Self Assessment UTS oleh AI

\hfill\break

\section{\texorpdfstring{\textbf{LAPORAN PENGUKURAN BERDASARKAN RUBRIK
DARI TUGAS
UTS}}{LAPORAN PENGUKURAN BERDASARKAN RUBRIK DARI TUGAS UTS}}\label{laporan-pengukuran-berdasarkan-rubrik-dari-tugas-uts}

\subsection{Identifikasi}\label{identifikasi}

\begin{enumerate}
\def\labelenumi{\arabic{enumi}.}
\tightlist
\item
  \textbf{Nama Mahasiswa dan NIM Penyusun TUGAS:} Muhammad Kinan
  Arkansyaddad (13523152)
\item
  \textbf{Nama Penilai:} Self Assessment oleh Gemini (Asisten AI)
\end{enumerate}

\begin{center}\rule{0.5\linewidth}{0.5pt}\end{center}

\subsection{Tinjauan Umum}\label{tinjauan-umum}

Karya UTS (1-4) yang diajukan oleh Muhammad Kinan Arkansyaddad
menunjukkan tingkat \textbf{orisinalitas, kohesi, dan wawasan diri
(insight)} yang luar biasa. Mahasiswa ini berhasil membangun sebuah
metafora sentral yang sangat kuat---``Hidup sebagai JRPG''---dan
mengembangkannya secara konsisten di keempat tugas.

Dimulai dengan ``All About Me'' (UTS-1) yang memperkenalkan \emph{Player
Character}, \emph{stats}, dan \emph{backstory} (Persona 4 Golden), tugas
ini kemudian diperkuat dengan ``Songs for You'' (UTS-2) yang memberikan
``lagu tema'' dari \emph{origin story} tersebut. Narasi ini berkembang
secara mendalam di ``My Stories for You'' (UTS-3) dengan memperkenalkan
konsep penyeimbang (\emph{Yotsuba\&!}), dan akhirnya diformalkan menjadi
sebuah analisis teknis dalam ``My Shape'' (UTS-4) sebagai ``Character
Build''.

Secara keseluruhan, ini adalah sebuah proyek yang dieksekusi dengan
sangat baik, kreatif, dan menunjukkan pemahaman mendalam tentang
bagaimana mengkomunikasikan pesan personal secara efektif.

\begin{center}\rule{0.5\linewidth}{0.5pt}\end{center}

\subsection{Tinjauan Spesifik}\label{tinjauan-spesifik}

Berikut adalah penilaian spesifik untuk setiap TUGAS berdasarkan rubrik
yang disediakan.

\subsubsection{\texorpdfstring{\textbf{UTS-1: All About
Me}}{UTS-1: All About Me}}\label{uts-1-all-about-me-1}

Presentasi ``All About Me'' menggunakan metafora ``Player Character'' di
``Level 3'' perkuliahan. Ini adalah cara yang sangat orisinal dan
efektif untuk memperkenalkan diri, lengkap dengan ``Character Sheet'',
``Job Skills'' (Game Dev), dan ``Backstory'' (Persona 4 Golden).

\begin{itemize}
\tightlist
\item
  \textbf{Orisinalitas (5/5):} Sangat unik. Jauh melampaui perkenalan
  standar dan menghadirkan sudut pandang yang sangat segar dan personal.
\item
  \textbf{Keterlibatan (5/5):} Sangat menarik. Metafora game membuat
  audiens (penilai) terus ingin tahu, dan penggunaan terminologi (Stats,
  Debuff, Quest) sangat cerdas.
\item
  \textbf{Humor (5/5):} Efektif dan relevan. Referensi seperti
  ``Difficulty `Dante Must Die'\,'', ``STAT (Stamina): 7 (Debuff
  `Begadang' kronis)'', dan ``Apa itu Bug? Maksudmu `Fitur' yang tak
  disengaja?'' sangat berhasil.
\item
  \textbf{Wawasan (Insight) (5/5):} Mendalam. Presentasi ini memberikan
  wawasan yang kuat tentang motivasi (\emph{backstory} P4G), ketakutan
  (``Fog of War'', ``Adulthood Debuff''), dan nilai-nilai yang dipegang.
\end{itemize}

\textbf{Skor (rata-rata): 5.0 / 5.0} \textbf{Saran Perbaikan:} Tidak
ada. Eksekusi sangat baik.

\subsubsection{\texorpdfstring{\textbf{UTS-2: My Songs for
You}}{UTS-2: My Songs for You}}\label{uts-2-my-songs-for-you-1}

Tugas ini menyajikan lagu ``Pursuing My True Self'' dari \emph{Persona 4
Golden}. Pilihan ini sangat relevan karena berfungsi sebagai ``lagu
tema'' untuk keseluruhan narasi yang dibangun di UTS-1.

\begin{itemize}
\tightlist
\item
  \textbf{Orisinalitas (5/5):} Sangat orisinal. Pilihan ini bukan
  sekadar ``lagu yang saya suka'', melainkan ``teks sumber'' dari
  seluruh metafora personal yang digunakan. Ini menunjukkan ikatan yang
  sangat unik.
\item
  \textbf{Keterlibatan (5/5):} Sangat memikat \emph{dalam konteks}. Bagi
  siapa pun yang mengikuti narasi dari UTS-1, pilihan lagu ini langsung
  ``klik'' dan terasa sangat berdampak.
\item
  \textbf{Humor (1/5):} Tidak ada. Rubrik mengharuskan penilaian humor,
  dan artefak (lagu dan lirik) yang disajikan tidak mengandung unsur
  humor. Skor ``1 - Buruk'' diberikan berdasarkan kriteria rubrik
  (``Tidak efektif atau tidak ada'').
\item
  \textbf{Inspirasi (5/5):} Sangat menginspirasi. Lirik lagu (``pursuing
  my true self'', ``We're all trapped in a maze of relationships'')
  selaras sempurna dengan wawasan di UTS-1 dan memberikan kesan
  mendalam.
\end{itemize}

\textbf{Skor (rata-rata): 4.0 / 5.0} \textbf{Saran Perbaikan:} Kriteria
``Humor'' sulit dipenuhi untuk lagu yang serius. Mungkin, menambahkan
satu slide pengantar singkat dengan analisis personal (mengapa lagu ini
dipilih) dapat membuat penyerahan tugas ini terasa lebih utuh sebagai
sebuah ``presentasi'' personal.

\subsubsection{\texorpdfstring{\textbf{UTS-3: My Stories for
You}}{UTS-3: My Stories for You}}\label{uts-3-my-stories-for-you-1}

Presentasi ini secara brilian mengembangkan narasi. Mahasiswa ini
memperkenalkan ``Game Guide'' kedua (\emph{Yotsuba\&!}) sebagai
penyeimbang dari ``Main Quest'' (\emph{Persona 4}). Ini adalah kisah
inspiratif tentang menyeimbangkan \emph{grinding} jangka panjang dengan
kebahagiaan sehari-hari.

\begin{itemize}
\tightlist
\item
  \textbf{Orisinalitas (5/5):} Pengembangan yang sangat segar.
  Memperkenalkan metafora \emph{kedua} untuk melengkapi yang pertama
  (Main Quest vs Daily Quest) adalah ide yang sangat unik.
\item
  \textbf{Keterlibatan (5/5):} Sangat memikat. Konsepnya mudah dipahami,
  relevan untuk audiens (mahasiswa `Level 3'), dan ditutup dengan
  ``Co-op Strategy'' yang kuat.
\item
  \textbf{Pengembangan Narasi (5/5):} Sempurna. Tugas ini menyambung
  langsung dari UTS-1 dan UTS-2, lalu menambahkan lapisan baru yang
  menunjukkan pertumbuhan dan pemikiran yang lebih matang.
\item
  \textbf{Inspirasi (5/5):} Sangat menginspirasi. Pesan intinya
  (``Jangan lupa menikmati \emph{gameplay} sederhana di tengah
  \emph{grinding} `Level 3' yang berat ini'') sangat kuat dan relevan.
\end{itemize}

\textbf{Skor (rata-rata): 5.0 / 5.0} \textbf{Saran Perbaikan:} Tidak
ada. Eksekusi sangat baik.

\subsubsection{\texorpdfstring{\textbf{UTS-4: My
Shape}}{UTS-4: My Shape}}\label{uts-4-my-shape-1}

Tugas ini berfungsi sebagai ``Player Stats Screen'' teknis, memformalkan
semua \emph{lore} dan \emph{insight} dari UTS-1, 2, dan 3 ke dalam
kerangka kerja (Identitas Naratif, Piagam Diri, VIA, dan SHAPE).

\begin{itemize}
\tightlist
\item
  \textbf{Orisinalitas (5/5):} Pengembangan yang sangat orisinal.
  Menggunakan kerangka kerja formal (SHAPE) dan menerjemahkannya
  sepenuhnya ke dalam metafora game (``Character Build Analysis'',
  ``Player Codex'') adalah langkah yang brilian.
\item
  \textbf{Keterlibatan (5/5):} Sangat memikat. Meskipun berisi laporan
  analitis, presentasi ini tidak pernah terasa kering karena konsisten
  menggunakan terminologi game yang sudah mapan.
\item
  \textbf{Pengembangan Narasi (5/5):} Sempurna. Ini adalah puncak dari
  pengembangan narasi, di mana semua ide abstrak dari tugas-tugas
  sebelumnya dikumpulkan, dianalisis, dan disajikan sebagai ``build''
  karakter yang utuh.
\item
  \textbf{Inspirasi (4/5):} Baik. Tugas ini lebih fokus pada
  \emph{analisis} daripada \emph{inspirasi} (dibandingkan UTS-3).
  Meskipun slide terakhir (``build ini Otentik'') inspiratif, sebagian
  besar isinya bersifat laporan.
\end{itemize}

\textbf{Skor (rata-rata): 4.75 / 5.0} \textbf{Saran Perbaikan:} Sangat
kuat. Untuk memaksimalkan poin ``Inspirasi'', sebuah slide kesimpulan
yang lebih kuat yang merangkum \emph{mengapa} ``build otentik'' ini
adalah sumber inspirasi dapat ditambahkan.

\subsubsection{\texorpdfstring{\textbf{UTS-5: My Personal
Reviews}}{UTS-5: My Personal Reviews}}\label{uts-5-my-personal-reviews-1}

\begin{itemize}
\tightlist
\item
  Self-Assessment menggunakan AI (Skor 2) ✅
\item
  Peer-Assessment 1 Tanpa AI (Skor 4) ✅
\item
  Peer-Assessment 2 Tanpa AI (Skor 4) ✅
\item
  Peer-Assessment 3 Tanpa AI (Skor 4, bonus) ✅
\end{itemize}

Assessment dapat dilihat di lembar skor di bawah Lembar Skor:
\href{UTS-5_skor.xlsx}{Download Lembar Skor UTS-5 (XLSX)}

\begin{center}\rule{0.5\linewidth}{0.5pt}\end{center}

\subsection{SKOR}\label{skor}

Berikut adalah rekapitulasi skor dan kontribusinya terhadap Capaian
Pembelajaran Mata Kuliah (CPMK) berdasarkan Tabel 1 dalam dokumen
panduan.

Skor (0-100) dihitung dari skor rubrik (rata-rata / 5 * 100). Kontribusi
CPMK dihitung dari (Skor * Bobot CPMK).

\begin{longtable}[]{@{}
  >{\raggedright\arraybackslash}p{(\linewidth - 10\tabcolsep) * \real{0.1429}}
  >{\raggedright\arraybackslash}p{(\linewidth - 10\tabcolsep) * \real{0.1429}}
  >{\centering\arraybackslash}p{(\linewidth - 10\tabcolsep) * \real{0.1786}}
  >{\centering\arraybackslash}p{(\linewidth - 10\tabcolsep) * \real{0.1786}}
  >{\centering\arraybackslash}p{(\linewidth - 10\tabcolsep) * \real{0.1786}}
  >{\centering\arraybackslash}p{(\linewidth - 10\tabcolsep) * \real{0.1786}}@{}}
\toprule\noalign{}
\begin{minipage}[b]{\linewidth}\raggedright
UTS
\end{minipage} & \begin{minipage}[b]{\linewidth}\raggedright
Skor (0-100)
\end{minipage} & \begin{minipage}[b]{\linewidth}\centering
Kontribusi CPMK-1
\end{minipage} & \begin{minipage}[b]{\linewidth}\centering
Kontribusi CPMK-2
\end{minipage} & \begin{minipage}[b]{\linewidth}\centering
Kontribusi CPMK-3
\end{minipage} & \begin{minipage}[b]{\linewidth}\centering
Kontribusi CPMK-4
\end{minipage} \\
\midrule\noalign{}
\endhead
\bottomrule\noalign{}
\endlastfoot
\textbf{UTS-1} & 100 & & 6.0 & & \\
\textbf{UTS-2} & 80 & & 5.6 & & \\
\textbf{UTS-3} & 100 & & 7.0 & & \\
\textbf{UTS-4} & 95 & & 5.7 & & \\
\textbf{UTS-5} & 100 & 10.0 & & & \\
\textbf{Total} & & \textbf{10.0} & \textbf{24.3} & \textbf{0.0} &
\textbf{0.0} \\
\end{longtable}

\bookmarksetup{startatroot}

\chapter{UAS-1: My Concepts (Konsep)}\label{uas-1-my-concepts-konsep}

\begin{center}\rule{0.5\linewidth}{0.5pt}\end{center}

\section{Latar Belakang}\label{latar-belakang}

Dari 10 masalah global terbesar, saya memilih fokus pada \textbf{Masalah
No.~8: Akses Pendidikan dan Melek Huruf}.

\textbf{Data global:}

\begin{itemize}
\tightlist
\item
  750 juta orang dewasa tidak dapat membaca atau menulis
\item
  250 juta anak tidak bersekolah
\item
  Metode konvensional gagal menjangkau mereka
\end{itemize}

Sebagai pengembang sistem informasi dan game, saya mengajukan
\textbf{solusi berbasis teknologi interaktif}.

\begin{center}\rule{0.5\linewidth}{0.5pt}\end{center}

\section{Definisi: Sistem Belajar Interaktif Berbasis
Tantangan}\label{definisi-sistem-belajar-interaktif-berbasis-tantangan}

\textbf{``Challenge-Based Interactive Learning System''} adalah konsep
yang saya tawarkan.

\textbf{Esensi:} Sebuah logika mesin abstrak yang mengumpulkan kekuatan
{[}K{]} untuk mengangkat beban masalah {[}B{]}. Intisari konsep ini
adalah \emph{``Kecanduan Positif''}---mengambil mekanisme psikologis
yang membuat orang betah bermain game (rasa penasaran, pencapaian,
progres visual) dan menyuntikkannya ke dalam proses belajar membaca.

\begin{center}\rule{0.5\linewidth}{0.5pt}\end{center}

\section{Dinamika Kekuatan vs Beban}\label{dinamika-kekuatan-vs-beban}

\subsection{Beban (The Burden)}\label{beban-the-burden}

\textbf{Rendahnya Motivasi dan Keterbatasan Akses Fisik}

\begin{itemize}
\tightlist
\item
  Kesenjangan literasi yang masif di Sub-Sahara Afrika dan Asia
\item
  Membangun sekolah fisik membutuhkan \textbf{miliaran dolar} dan
  \textbf{puluhan tahun}
\item
  Orang dewasa yang buta huruf merasa malu untuk kembali duduk di bangku
  sekolah dasar
\item
  Hambatan geografis dan ekonomi yang fundamental
\end{itemize}

\subsection{Kekuatan (The Force)}\label{kekuatan-the-force}

\textbf{Ubiquitas Smartphone dan Daya Tarik Media Interaktif}

\begin{itemize}
\tightlist
\item
  Hampir setiap rumah tangga, bahkan di negara berkembang, memiliki
  akses perangkat seluler
\item
  Game memiliki kekuatan unik: pemain \textbf{gagal tanpa dihakimi},
  justru \textbf{termotivasi mencoba lagi} (\emph{retry})
\item
  Ini adalah kekuatan mental yang kita perlukan untuk memberantas buta
  huruf
\item
  Distribusi instan tanpa infrastruktur fisik
\end{itemize}

\begin{center}\rule{0.5\linewidth}{0.5pt}\end{center}

\section{Turunan Ide Praktis}\label{turunan-ide-praktis}

Dari konsep induk ini, lahir beberapa ide yang dapat dieksekusi:

\subsection{Game RPG Edukasi}\label{game-rpg-edukasi}

Permainan petualangan di mana pemain \textbf{harus membaca instruksi}
untuk menyelesaikan misi, bukan sekadar menekan tombol.

\subsection{Kompetisi Desa Digital}\label{kompetisi-desa-digital}

Menggunakan papan skor (\emph{leaderboard}) lokal untuk memacu semangat
belajar antar-tetangga melalui aplikasi.

\subsection{NPC Guru Virtual}\label{npc-guru-virtual}

Karakter dalam game yang diprogram dengan AI untuk membimbing pemain
secara personal, menggantikan peran guru privat yang mahal.

\begin{center}\rule{0.5\linewidth}{0.5pt}\end{center}

\section{Kesimpulan}\label{kesimpulan}

Konsep ``Challenge-Based Interactive Learning System'' adalah
\textbf{logika abstrak} yang bekerja untuk mengubah krisis literasi
global menjadi peluang distribusi pendidikan yang demokratis melalui
teknologi yang sudah ada di tangan jutaan orang.

\bookmarksetup{startatroot}

\chapter{UAS-2: My Opinions (Opini)}\label{uas-2-my-opinions-opini}

\begin{center}\rule{0.5\linewidth}{0.5pt}\end{center}

\section{Mengubah ``Buang Waktu'' Menjadi Masa Depan
Pendidikan}\label{mengubah-buang-waktu-menjadi-masa-depan-pendidikan}

\begin{quote}
\textbf{``Video game adalah perpustakaan paling efektif di abad ke-21
yang belum dimanfaatkan sepenuhnya.''}
\end{quote}

Seringkali, video game dianggap sebagai aktivitas membuang waktu atau
merusak otak. Opini saya berdiri di sisi sebaliknya.

\begin{center}\rule{0.5\linewidth}{0.5pt}\end{center}

\section{Urgensi Masalah}\label{urgensi-masalah}

\textbf{Masalah literasi global sangat mendesak:}

\begin{itemize}
\tightlist
\item
  \textbf{10\% populasi dunia} hidup tanpa kemampuan dasar membaca
\item
  Ini memenjara mereka dalam \textbf{kemiskinan}
\item
  Tidak dapat mengakses: informasi kesehatan, kontrak kerja, hak sipil
\end{itemize}

\textbf{Status quo tidak efektif:} Metode ceramah satu arah di ruang
kelas padat \textbf{terbukti tidak cukup cepat} untuk mengejar
ketertinggalan.

\begin{center}\rule{0.5\linewidth}{0.5pt}\end{center}

\section{Argumen Utama}\label{argumen-utama}

\subsection{1. Pendidikan vs Hiburan: False
Dichotomy}\label{pendidikan-vs-hiburan-false-dichotomy}

\textbf{Kita harus berhenti melihat keduanya sebagai kutub berlawanan.}

Anak zaman sekarang adalah \emph{digital natives}. Memaksa belajar
dengan metode abad ke-19 = \textbf{resep kegagalan}.

\subsection{2. Bukti dari Game
Komersial}\label{bukti-dari-game-komersial}

Game seperti \emph{Zelda} atau \emph{Minecraft} membuktikan: - Anak-anak
rela \textbf{menghafal ratusan resep} \emph{crafting} - Membaca
\textbf{teks dialog yang rumit} demi mencapai tujuan - Ini adalah
\textbf{motivasi intrinsik}, bukan paksaan

\subsection{3. Visi Alternatif}\label{visi-alternatif}

Jika kita \textbf{menyusupkan kurikulum literasi} ke format yang sama
menariknya:

\begin{itemize}
\tightlist
\item
  ❌ Kita \textbf{tidak perlu ``menyuruh''} anak belajar
\item
  ✅ Mereka akan \textbf{memintanya sendiri}
\end{itemize}

\subsection{4. Strategi Akses}\label{strategi-akses}

Ini bukan \textbf{menggantikan sekolah}, tetapi \textbf{menciptakan
jalan pintas} (\emph{bypass}) untuk yang tertinggal: - Teknologi
distribusi instan tanpa infrastruktur fisik - ``Sekolah'' bisa
ditebarkan ke \textbf{saku celana miliaran orang} dalam detik

\begin{center}\rule{0.5\linewidth}{0.5pt}\end{center}

\section{Kesimpulan}\label{kesimpulan-1}

\textbf{Mengabaikan potensi ini di tengah krisis literasi global adalah
kelalaian intelektual.}

Game bukan musuh pendidikan. \textbf{Game adalah instrumen transformasi
terpotensial di abad ini.}

\bookmarksetup{startatroot}

\chapter{UAS-3: My Innovations
(Inovasi)}\label{uas-3-my-innovations-inovasi}

\begin{center}\rule{0.5\linewidth}{0.5pt}\end{center}

\section{``Aksara Quest'' - Petualangan Literasi Tanpa
Batas}\label{aksara-quest---petualangan-literasi-tanpa-batas}

Inovasi ini adalah \textbf{produk nyata berupa perangkat lunak
permainan} yang dirancang untuk mengajarkan membaca dan berhitung dasar
\textbf{tanpa koneksi internet}.

\begin{center}\rule{0.5\linewidth}{0.5pt}\end{center}

\section{Deskripsi Solusi Teknis}\label{deskripsi-solusi-teknis}

Berbeda dengan aplikasi belajar biasa (Duolingo, Ruangguru) yang terasa
seperti ``buku pelajaran digital'', \textbf{Aksara Quest} adalah
\textbf{game RPG penuh aksi}.

\subsection{Stack Teknologi}\label{stack-teknologi}

\begin{longtable}[]{@{}ll@{}}
\toprule\noalign{}
Aspek & Detail \\
\midrule\noalign{}
\endhead
\bottomrule\noalign{}
\endlastfoot
\textbf{Platform} & Android (Entry Level spec) \\
\textbf{Engine} & Godot Engine (ringan, open source) \\
\textbf{Bahasa} & GDScript + C++ (GDExtension) \\
\end{longtable}

\begin{center}\rule{0.5\linewidth}{0.5pt}\end{center}

\section{Fitur Unggulan (Value
Innovation)}\label{fitur-unggulan-value-innovation}

Inovasi nilai terletak pada cara menghilangkan rasa \textbf{``sedang
belajar''}:

\subsection{Mekanisme Mantra Suara (Voice-Activated
Spells)}\label{mekanisme-mantra-suara-voice-activated-spells}

\begin{itemize}
\tightlist
\item
  Pemain mengalahkan monster dengan \textbf{mengucapkan kata dengan
  benar} (bukan tombol)
\item
  Sistem pengenalan suara \emph{offline} memvalidasi ucapan
\item
  Melatih \textbf{pelafalan dan membaca cepat} dalam situasi yang seru
\end{itemize}

\begin{center}\rule{0.5\linewidth}{0.5pt}\end{center}

\section{Dampak Nyata}\label{dampak-nyata}

Inovasi ini secara langsung menjawab tantangan \textbf{Masalah Global
\#8} dengan spesifikasi:

\begin{longtable}[]{@{}ll@{}}
\toprule\noalign{}
Metrik & Target \\
\midrule\noalign{}
\endhead
\bottomrule\noalign{}
\endlastfoot
\textbf{Ukuran Aplikasi} & \textless{} 100MB \\
\textbf{Kompatibilitas} & HP lama (RAM 2GB+) \\
\textbf{Konektivitas} & 100\% Offline-capable \\
\textbf{Target Audiens} & Daerah 3T, tanpa guru/sekolah \\
\end{longtable}

\textbf{Dampak:} Mendemokratisasi akses pendidikan berkualitas tinggi ke
jutaan orang di daerah terpencil.

\begin{center}\rule{0.5\linewidth}{0.5pt}\end{center}

\section{Kesimpulan}\label{kesimpulan-2}

\emph{Aksara Quest} adalah gerakan sosial yang memberdayakan melalui
\textbf{game design yang terukur} dan \textbf{engineering yang solid},
bukan sekadar aplikasi edukasi biasa - Semakin tinggi level literasi,
semakin kompleks cerita yang terbuka - Gamifikasi progres pembelajaran

\subsection{🤖 AI Lokal yang Ringan}\label{ai-lokal-yang-ringan}

\begin{itemize}
\tightlist
\item
  \emph{Small Language Model} (SLM) dioptimalkan dengan C++
\item
  NPC dapat diajak mengobrol secara dinamis
\item
  Melatih \textbf{pemahaman konteks (\emph{reading comprehension})}
\item
  \textbf{Tanpa perlu internet} untuk cloud server
\end{itemize}

\subsection{3. Dampak Nyata}\label{dampak-nyata-1}

Inovasi ini secara langsung menjawab tantangan Masalah Global \#8.
Dengan ukuran aplikasi di bawah 100MB dan kemampuan berjalan di HP lama,
\emph{Aksara Quest} mendemokratisasi akses pendidikan berkualitas tinggi
ke daerah terpencil (3T) yang tidak memiliki guru atau sekolah fisik.

\bookmarksetup{startatroot}

\chapter{UAS-4: My Knowledge
(Pengetahuan)}\label{uas-4-my-knowledge-pengetahuan}

\begin{center}\rule{0.5\linewidth}{0.5pt}\end{center}

\section{Pengenalan}\label{pengenalan}

Sebagai \emph{engineer} sistem informasi, pengetahuan saya mencakup
tidak hanya ide, tetapi juga \textbf{struktur teknis} untuk
mewujudkannya. Pengetahuan dipetakan ke dalam \textbf{tiga kategori}:

\begin{center}\rule{0.5\linewidth}{0.5pt}\end{center}

\section{Peta Pengetahuan Primitif (The
``What'')}\label{peta-pengetahuan-primitif-the-what}

\textbf{Fondasi pengetahuan deklaratif} yang harus dikuasai:

\subsection{Game Development Core}\label{game-development-core}

\begin{itemize}
\tightlist
\item
  Pemahaman mendalam tentang \emph{Scene Tree}, \emph{Signal}, dan
  \emph{Physics Process} di Godot Engine
\item
  Memahami game loop, event-driven programming, dan resource management
\end{itemize}

\subsection{Linguistik Komputasional}\label{linguistik-komputasional}

\begin{itemize}
\tightlist
\item
  Dasar-dasar pemrosesan teks dan suara manusia oleh komputer
\item
  Konsep Phoneme, Tokenization, NLP fundamentals
\item
  Speech recognition offline dan text-to-speech
\end{itemize}

\subsection{Statistik Global}\label{statistik-global}

\begin{itemize}
\tightlist
\item
  Data target audiens: wilayah buta huruf, penetrasi smartphone di
  negara berkembang
\item
  Demografi pengguna mobile entry-level, ketersediaan bandwidth
\end{itemize}

\begin{center}\rule{0.5\linewidth}{0.5pt}\end{center}

\section{Peta Pemecahan Masalah (The
``How'')}\label{peta-pemecahan-masalah-the-how}

\textbf{Pengetahuan prosedural} tentang \emph{workflow} pembuatan
inovasi:

\subsection{Fase Perancangan (Design)}\label{fase-perancangan-design}

\begin{itemize}
\tightlist
\item
  Menerjemahkan kurikulum SD kelas 1 menjadi mekanik \emph{gameplay}
\item
  \textbf{Contoh:} ``Subjek-Predikat-Objek'' → misi ``Menyusun Jembatan
  Kata''
\item
  Game loop design, level progression, feedback system
\end{itemize}

\subsection{Fase Optimasi
(Optimization)}\label{fase-optimasi-optimization}

\begin{itemize}
\tightlist
\item
  Memangkas penggunaan RAM agar berjalan lancar di perangkat 2GB
\item
  Pemrograman C++ tingkat rendah, memory profiling
\item
  Asset compression, batching, culling techniques
\end{itemize}

\subsection{Fase Distribusi}\label{fase-distribusi}

\begin{itemize}
\tightlist
\item
  Strategi penyebaran \emph{offline} (peer-to-peer sharing) di desa
  tanpa internet
\item
  APK distribution, USB-based sharing, sync mechanisms
\end{itemize}

\begin{center}\rule{0.5\linewidth}{0.5pt}\end{center}

\section{Artefak Pengetahuan}\label{artefak-pengetahuan}

\textbf{Bukti nyata kompetensi:}

\begin{longtable}[]{@{}
  >{\raggedright\arraybackslash}p{(\linewidth - 2\tabcolsep) * \real{0.4737}}
  >{\raggedright\arraybackslash}p{(\linewidth - 2\tabcolsep) * \real{0.5263}}@{}}
\toprule\noalign{}
\begin{minipage}[b]{\linewidth}\raggedright
Artefak
\end{minipage} & \begin{minipage}[b]{\linewidth}\raggedright
Deskripsi
\end{minipage} \\
\midrule\noalign{}
\endhead
\bottomrule\noalign{}
\endlastfoot
\textbf{GDD} & Cetak biru 50 halaman merinci setiap aspek permainan \\
\textbf{Prototipe Alpha} & Versi playable dengan satu level tutorial \\
\textbf{Video Devlog} & Dokumentasi visual proses pengembangan untuk
komunitas \\
\end{longtable}

\begin{center}\rule{0.5\linewidth}{0.5pt}\end{center}

\section{Sintesis: Mengintegrasikan Ketiga
Pilar}\label{sintesis-mengintegrasikan-ketiga-pilar}

\begin{itemize}
\tightlist
\item
  \textbf{What} → Fondasi teoretis dan konseptual
\item
  \textbf{How} → Jalan eksekusi yang terukur
\item
  \textbf{Artifacts} → Bukti nyata dan terukur
\end{itemize}

\textbf{Kombinasi ketiganya adalah satu-satunya cara untuk mewujudkan
\emph{Aksara Quest} yang berdampak sosial.}

\bookmarksetup{startatroot}

\chapter{UAS-5 My Professional
Reviews}\label{uas-5-my-professional-reviews}

Berikut adalah hasil evaluasi mandiri (\emph{Self-Assessment}) terhadap
kualitas pengerjaan UAS-1 hingga UAS-4.

📊 \textbf{Download Scoring Sheet:}
\href{../My_Professional_Reviews/UAS-5_skor.xlsx}{UAS-5\_skor.xlsx}

\section{1. Evaluasi Konsep (UAS-1)}\label{evaluasi-konsep-uas-1}

\begin{longtable}[]{@{}
  >{\raggedright\arraybackslash}p{(\linewidth - 4\tabcolsep) * \real{0.3333}}
  >{\raggedright\arraybackslash}p{(\linewidth - 4\tabcolsep) * \real{0.3333}}
  >{\raggedright\arraybackslash}p{(\linewidth - 4\tabcolsep) * \real{0.3333}}@{}}
\toprule\noalign{}
\begin{minipage}[b]{\linewidth}\raggedright
Kriteria
\end{minipage} & \begin{minipage}[b]{\linewidth}\raggedright
Skor
\end{minipage} & \begin{minipage}[b]{\linewidth}\raggedright
Justifikasi Penilaian
\end{minipage} \\
\midrule\noalign{}
\endhead
\bottomrule\noalign{}
\endlastfoot
\textbf{Clarity (Kejelasan)} & \textbf{5} & Konsep ``Sistem Belajar
Interaktif Berbasis Tantangan'' didefinisikan dengan struktur yang
sangat jernih. Perbedaan antara elemen abstrak (logika mesin permainan)
dan konkret (implementasi RPG) dijelaskan tanpa ambiguitas. \\
\textbf{Logic (Logika)} & \textbf{5} & Alur logika sangat koheren:
Menggunakan kekuatan psikologis \emph{game} (adiksi positif,
\emph{reward}) untuk mengangkat beban spesifik (kurangnya motivasi \&
akses sekolah fisik). Hubungan sebab-akibat terbangun dengan kuat. \\
\textbf{Validity (Validitas)} & \textbf{5} & Konsep didukung oleh data
faktual mengenai krisis literasi global (750 juta buta huruf) dan tren
adopsi teknologi \emph{mobile}, menjadikannya solusi yang valid secara
empiris. \\
\textbf{Usefulness (Kegunaan)} & \textbf{5} & Konsep ini menawarkan
manfaat praktis yang signifikan bagi daerah 3T (Tertinggal, Terdepan,
Terluar) yang tidak terjangkau infrastruktur pendidikan konvensional. \\
\end{longtable}

\section{2. Evaluasi Opini (UAS-2)}\label{evaluasi-opini-uas-2}

\begin{longtable}[]{@{}
  >{\raggedright\arraybackslash}p{(\linewidth - 4\tabcolsep) * \real{0.3333}}
  >{\raggedright\arraybackslash}p{(\linewidth - 4\tabcolsep) * \real{0.3333}}
  >{\raggedright\arraybackslash}p{(\linewidth - 4\tabcolsep) * \real{0.3333}}@{}}
\toprule\noalign{}
\begin{minipage}[b]{\linewidth}\raggedright
Kriteria
\end{minipage} & \begin{minipage}[b]{\linewidth}\raggedright
Skor
\end{minipage} & \begin{minipage}[b]{\linewidth}\raggedright
Justifikasi Penilaian
\end{minipage} \\
\midrule\noalign{}
\endhead
\bottomrule\noalign{}
\endlastfoot
\textbf{Compelling (Menarik)} & \textbf{5} & Judul dan pembukaan esai
(``Video Game sebagai Perpustakaan Abad 21'') sangat memikat dan
langsung menangkap perhatian pembaca dengan provokasi intelektual yang
kuat. \\
\textbf{Informatif} & \textbf{5} & Opini tidak hanya berisi keluhan,
tetapi kaya akan informasi data statistik literasi global yang relevan
dan wawasan tentang perilaku \emph{digital natives}. \\
\textbf{Persuasif} & \textbf{5} & Argumen yang dibangun berhasil
meyakinkan pembaca bahwa metode lama telah gagal dan gamifikasi adalah
solusi logis yang tak terelakkan, didukung premis yang solid. \\
\textbf{Engaging (Melibatkan)} & \textbf{5} & Tulisan menyentuh sisi
emosional dan intelektual pembaca, mengajak mereka melihat potensi
anak-anak yang sering disalahpahami sebagai ``pemalas'' menjadi
pembelajar potensial. \\
\end{longtable}

\section{3. Evaluasi Inovasi (UAS-3)}\label{evaluasi-inovasi-uas-3}

\begin{longtable}[]{@{}
  >{\raggedright\arraybackslash}p{(\linewidth - 4\tabcolsep) * \real{0.3333}}
  >{\raggedright\arraybackslash}p{(\linewidth - 4\tabcolsep) * \real{0.3333}}
  >{\raggedright\arraybackslash}p{(\linewidth - 4\tabcolsep) * \real{0.3333}}@{}}
\toprule\noalign{}
\begin{minipage}[b]{\linewidth}\raggedright
Kriteria
\end{minipage} & \begin{minipage}[b]{\linewidth}\raggedright
Skor
\end{minipage} & \begin{minipage}[b]{\linewidth}\raggedright
Justifikasi Penilaian
\end{minipage} \\
\midrule\noalign{}
\endhead
\bottomrule\noalign{}
\endlastfoot
\textbf{Guna (Kegunaan)} & \textbf{5} & Inovasi ``Aksara Quest''
menjawab kebutuhan mendesak akan alat bantu literasi mandiri yang tidak
bergantung pada kehadiran guru fisik. \\
\textbf{Kebaruan (Novelty)} & \textbf{5} & Penggabungan mekanisme
``Mantra Suara'' (\emph{Voice-Activated Spells}) dengan teknologi
\emph{Small Language Model} (SLM) \emph{offline} di perangkat
\emph{low-end} adalah terobosan baru dalam \emph{EdTech}. \\
\textbf{Desain} & \textbf{5} & Rancangan teknis sangat matang, memilih
\emph{Godot Engine} dan C++ untuk memastikan performa maksimal di
perangkat spesifikasi rendah, menunjukkan pemahaman desain sistem yang
dalam. \\
\textbf{Dampak} & \textbf{5} & Memiliki potensi dampak sosial yang masif
(skalabilitas tinggi) karena aplikasi dapat didistribusikan secara
viral/P2P tanpa infrastruktur server yang mahal. \\
\end{longtable}

\section{4. Evaluasi Pengetahuan
(UAS-4)}\label{evaluasi-pengetahuan-uas-4}

\begin{longtable}[]{@{}
  >{\raggedright\arraybackslash}p{(\linewidth - 4\tabcolsep) * \real{0.3333}}
  >{\raggedright\arraybackslash}p{(\linewidth - 4\tabcolsep) * \real{0.3333}}
  >{\raggedright\arraybackslash}p{(\linewidth - 4\tabcolsep) * \real{0.3333}}@{}}
\toprule\noalign{}
\begin{minipage}[b]{\linewidth}\raggedright
Kriteria
\end{minipage} & \begin{minipage}[b]{\linewidth}\raggedright
Skor
\end{minipage} & \begin{minipage}[b]{\linewidth}\raggedright
Justifikasi Penilaian
\end{minipage} \\
\midrule\noalign{}
\endhead
\bottomrule\noalign{}
\endlastfoot
\textbf{Kurasi} & \textbf{5} & Pemilihan topik pengetahuan (Game Engine,
Linguistik Komputasional, Statistik Global) sangat tepat sasaran dan
relevan untuk mendukung inovasi yang diajukan. \\
\textbf{Kejelasan} & \textbf{5} & Peta pengetahuan dipisahkan dengan
jelas antara pengetahuan deklaratif (Primitif) dan prosedural (Pemecahan
Masalah), memudahkan pembaca memahami struktur berpikir. \\
\textbf{Akurasi} & \textbf{5} & Istilah teknis seperti \emph{Scene
Tree}, \emph{GDExtension}, dan \emph{Phoneme} digunakan secara akurat
sesuai konteks rekayasa perangkat lunak dan linguistik. \\
\textbf{Daya Guna} & \textbf{5} & Artefak pengetahuan yang dihasilkan
(GDD, Prototipe, Devlog) sangat aplikatif dan bisa langsung digunakan
sebagai panduan pengembangan nyata. \\
\end{longtable}

\begin{center}\rule{0.5\linewidth}{0.5pt}\end{center}

\textbf{Kesimpulan Skor Akhir:} Berdasarkan penilaian di atas, rata-rata
skor keseluruhan adalah \textbf{5.0}. Penilaian ini mencerminkan
keyakinan penuh terhadap kualitas, kedalaman, dan relevansi solusi yang
ditawarkan dalam menjawab Masalah Global No.~8.

\bookmarksetup{startatroot}

\chapter{Summary}\label{summary}

In summary, this book has no content whatsoever.

\bookmarksetup{startatroot}

\chapter*{References}\label{references}
\addcontentsline{toc}{chapter}{References}

\markboth{References}{References}

\phantomsection\label{refs}




\end{document}
